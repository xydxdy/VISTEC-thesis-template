\chapter{Introduction}
\label{chapter1}





\section{Heading}


\paragraph{The chapter headings should be 14 points and any other titles should be in 12 points.  The text in the chapter body should be computer printed in 12 points Times New Roman font.}



\subsection{Sub-heading 1}

\subparagraph{
Typing should be with a spacing of 1.5 between lines, including the List of References and Appendices.
}






\subsubsection{Sub-heading 2}

\subsubparagraph{
\lipsum[1][1-3] % dummy text
}

\begin{enumerate}
\item Enumerate One
\item Enumerate Two
\item Enumerate Three
\end{enumerate}





\section{Algorithm}

\paragraph{This is an example of \autoref{chap1:algo:my-algo}.}

\begin{algorithm}[ht]
\caption{An algorithm with caption.}
\label{chap1:algo:my-algo}
\normalsize\singlespacing
\begin{algorithmic}[1] % [1] print number all lines 
    \Require $n \geq 0$
    \Ensure $y = x^n$
    \State $y \gets 1$
    \State $X \gets x$
    \State $N \gets n$
    \While{$N \neq 0$}
    \If{$N$ is even}
        \State $X \gets X \times X$
        \State $N \gets \frac{N}{2}$  \Comment{This is a comment}
    \ElsIf{$N$ is odd}
        \State $y \gets y \times X$
        \State $N \gets N - 1$
    \EndIf
    \EndWhile
\end{algorithmic}
\end{algorithm}



\section{Equation} 

\paragraph{As an illustration of \LaTeX's mathematics formatting,
\autoref{chap1:eq:renyi} is the definition of {\em R\'enyi entropy} and \autoref{chap1:eq:total-loss} is the total loss function:
}

%%%%%%%%%%%%%%%%%%%%%%%%% EQUATION  %%%%%%%%%%%%%%%%%%%%%%%%% 
\begin{equation}
\label{chap1:eq:renyi}
H_{\alpha}(X) =
\frac{1}{1-\alpha}
\log \left(\sum_{x \in {\cal X}}P[X=x]^{\alpha} \right) .
\end{equation}

%%%%%%%%%%%%%%%%%%%%%%%%%  EQUATION %%%%%%%%%%%%%%%%%%%%%%%%% 
\begin{equation} 
\label{chap1:eq:total-loss}
\begin{aligned}
\mathcal{L}_{\textrm{total}} = \frac{1}{N}\sum_{i=1}^{N}\{w_i\mathcal{L}_i\}. 
\end{aligned}
\end{equation}


\paragraph{
\lipsum[1][1-3] % dummy text
}