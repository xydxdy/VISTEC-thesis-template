%----------------------------------------
\chapter{VISTEC Thesis Formatting}
\label{chapter4}
%----------------------------------------

%========================
\section{Overview}
%========================
\begin{paragraph}
This chapter presents examples of standardized formatting for a VISTEC thesis, including guidelines for headings, equations, algorithms, tables, figures, citations, and footnotes. Each example demonstrates the intended structure and style to ensure consistency throughout the document.
\end{paragraph}

%========================
\section{Headings}
%========================
\begin{paragraph}
This section provides an example of a paragraph placed under a main section heading. It is used to introduce and briefly describe the topic or content area that will be elaborated upon in the following subsections.
\end{paragraph}

%========================
\subsection{Subheadings}
%========================
\begin{subparagraph}
This subsection demonstrates the formatting for subheadings. Text under a subheading serves to further detail specific aspects of the main section, offering a more focused discussion within the broader topic.
\end{subparagraph}


\subsubsection{Second-Level Subheading}
\begin{subsubparagraph}
This is a subsubparagraph under the second-level subheading. It is typically used for listing or elaborating fine-grained points.
\end{subsubparagraph}

\begin{enumerate}[itemindent=\subsubparitemindent]
    \item This is the first item in the enumerated list.
    \item This is the second item in the enumerated list.
    \item This is the third item in the enumerated list.
\end{enumerate}

\begin{subsubparagraph}
This subsubparagraph provides additional commentary or explanation following the enumerated list.
\end{subsubparagraph}
    

%========================
\section{Equations}
%========================
\begin{paragraph}
The following is an example of formatting mathematical equations. As illustrated in \autoref{ch4:eq:renyi}, the {\em R\'enyi entropy} is defined as:
\end{paragraph}

\begin{equation}
\label{ch4:eq:renyi}
H_{\alpha}(X) =
\frac{1}{1-\alpha}
\log \left(\sum_{x \in {\cal X}}P[X=x]^{\alpha} \right) .
\end{equation}

%========================
\section{Algorithms}
%========================
\begin{paragraph}
Algorithms can be presented using the \texttt{algorithmic} package, as shown in \autoref{ch4:algo:my-algo}.
\end{paragraph}

\begin{algorithm}[h]
\caption{An example algorithm with a caption.}
\label{ch4:algo:my-algo}
\normalsize\singlespacing
\begin{algorithmic}[1]
    \Require $n \geq 0$
    \Ensure $y = x^n$
    \State $y \gets 1$
    \State $X \gets x$
    \State $N \gets n$
    \While{$N \neq 0$}
        % \If{$N$ is even}
            \State $X \gets X \times X$
            \State $N \gets \frac{N}{2}$ \Comment{example comment}
        % \ElsIf{$N$ is odd}
        %     \State $y \gets y \times X$
        %     \State $N \gets N - 1$
        % \EndIf
    \EndWhile
\end{algorithmic}
\end{algorithm}

%========================
\section{Tables}
%========================
\begin{paragraph}
{\LaTeX} table generators, such as \href{https://www.tablesgenerator.com/}{TablesGenerator.com}, can help you easily create well-formatted tables. See \autoref{ch4:table:results} for an example.
\end{paragraph}

\begin{table}[ht]
\caption{Classification performance. An asterisk ($^*$) indicates statistically significant results ($p<0.05$).}
\label{ch4:table:results}
\centering
\normalsize\singlespacingplus
    \begin{tabular}{@{}ccc@{}}
    \toprule
    \multirow{2}{*}{\textbf{Comparison Model}} & \multicolumn{2}{c}{\textbf{Subject-independent}}       \\ \cmidrule(l){2-3} 
                                               & \textbf{Accuracy $\pm$ SD}          & \textbf{F1-score $\pm$ SD}         \\ \midrule
    FBCSP-SVM                                  & $64.96 \pm 12.70$          & $65.25 \pm 15.14$         \\
    Deep Convnet                               & $68.33 \pm 15.33$          & $70.20 \pm 15.18$         \\
    EEGNet-8,2                                 & $68.84 \pm 13.87$          & $70.39 \pm 14.30$         \\
    Spectral-Spatial CNN                       & $68.27 \pm 13.56$          & $65.86 \pm 17.37$         \\
    MIN2Net                                    & $\mathbf{72.03 \pm 14.04^*}$ & $\mathbf{72.62 \pm 14.14^*}$ \\ \bottomrule
    \end{tabular}%
\end{table}

%========================
\section{Figures}
%========================
\begin{paragraph}
Figures can be included easily using the \texttt{graphicx} package. Example shown in \autoref{ch4:fig:fig-A} and \autoref{ch4:fig:mychemfig}.
\end{paragraph}

\begin{figure}[ht]
    \centering
    \includegraphics[width=1\columnwidth]{figures/ch4/A.pdf}
    \longcaption[Example figure with long caption.]{This figure demonstrates how to include a standard image (e.g., PDF, PNG, JPG) into your document. Long captions should be aligned properly.}
    \label{ch4:fig:fig-A}
\end{figure}

\begin{figure}[ht]
    \centering
    \chemnameinit{\chemfig{R-C(-[:-30]OH)=[:30]O}}
    \schemestart
        \chemname{\chemfig{R’OH}}{Alcohol}
        \+
        \chemname{\chemfig{R-C(-[:-30]OH)=[:30]O}}{Carboxylic acid}
        \arrow(.mid east--.mid west)
        \chemname{\chemfig{R-C(-[:-30]OR’)=[:30]O}}{Ester}
        \+
        \chemname{\chemfig{H_2O}}{Water}
    \schemestop
    \chemnameinit{}
    \caption{An esterification reaction illustrated using the \texttt{chemfig} package.}
    \label{ch4:fig:mychemfig}
\end{figure}

%========================
\section{Citations}
%========================
\begin{paragraph}
To cite references, use \verb|\cite{}|, such as \cite{min2net}, or multiple sources like \cite{hu79, somework2020, tonio_paper}. Ensure that the corresponding BibTeX entries are added to the \texttt{bibliography.bib} file before citing. Below is an example BibTeX entry:
\end{paragraph}

\begin{verbatim}
@ARTICLE{dummy2022example,
  author  = {Doe, John and Smith, Jane and Roe, Richard},
  journal = {Journal of Example Studies},
  title   = {A Dummy Title for Demonstration Purposes},
  year    = {2022},
  volume  = {42},
  number  = {1},
  pages   = {1--10}
}
\end{verbatim}

%========================
\section{Footnotes}
%========================
\begin{paragraph}
You can insert a footnote marker using \verb|\footnotemark|\footnotemark{} and define the text later with \verb|\footnotetext{Example footnote.}|\footnotetext{Example footnote.}
\end{paragraph}