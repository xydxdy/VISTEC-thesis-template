%----------------------------------------
\chapter{Introduction}
\label{chapter1}
%----------------------------------------

%========================
\section{Motivating Problem}
\label{ch1:sec:motivation}
%========================
\begin{paragraph}
This section describes the research motivation that forms the foundation of the thesis.
\end{paragraph}

%========================
\section{Contributions}
\label{ch1:sec:Contributions}
%========================
\begin{paragraph}
This thesis makes the following key contributions:
\end{paragraph}

\begin{itemize}[leftmargin=\paritemindent]
    \item We introduce a novel experimental paradigm that addresses key limitations in the current research.
    \item We propose a novel algorithm that enhances learning performance across multiple tasks.
\end{itemize}

%========================
\section{Outline}
\label{ch1:sec:outline}
%========================
\begin{paragraph}
This thesis is organized into the following chapters:

\textbf{\autoref{chapter1} Introduction:}  
Introduces the research motivation, key contributions, and provides an overview of the thesis structure.

\textbf{\autoref{chapter2} How to Use This Template:}  
Provides practical guidance on how to work with and customize the VISTEC \LaTeX{} thesis template for efficient thesis writing.

\textbf{\autoref{chapter3} LaTeX Usage Examples:}  
Demonstrates how to format and organize content in \LaTeX{} through examples of paragraphs, sections, equations, algorithms, tables, figures, citations, and footnotes.

\textbf{\autoref{chapter4} Investigation:}  
Describes the investigation, expanding upon the results and analysis from the previous study to validate the proposed approaches.

\textbf{\autoref{chapter5} Conclusion:}  
Summarizes the major findings, discusses their implications, and suggests future research directions.

\textbf{\autoref{appendix} Proofs Supporting Investigation:}  
Presents supplementary materials, including detailed proofs, additional results, and extended discussions that support the main chapters.
\end{paragraph}
