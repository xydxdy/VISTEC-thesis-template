\documentclass{VISTEC}


\usepackage[most]{tcolorbox}
\usepackage{array}

\renewcommand{\thesection}{\arabic{section}}

\newgeometry{
        left=2.5cm,
        right=2.5cm,
        top=2.5cm,
        bottom=2.5cm,
    }

\begin{document} 

%----------------------------------------
\chapter*{How to Use This Template}
\label{manual}
%----------------------------------------

%========================
\section{Getting Started with {\LaTeX}}
\label{manual:latex-getting-started}
%========================

\begin{paragraph}
If you are new to {\LaTeX}, start by learning the basics such as document structure, environments, referencing, and how to compile a file. A good place to begin is the Overleaf Learn website:
\end{paragraph}

\begin{center}
\href{https://www.overleaf.com/learn}{\texttt{https://www.overleaf.com/learn}}
\end{center}




%========================
\section{Download and Installation}
\label{manual:download}
%========================
\begin{paragraph}
\begin{enumerate}[leftmargin=\paritemindent, itemsep=6pt]
\item To use this template, download the \texttt{VISTEC\_Thesis\_Template.zip} file from the GitHub repository: \href{https://github.com/xydxdy/VISTEC-thesis-template/releases}{\texttt{https://github.com/xydxdy/VISTEC-thesis-template/releases}}.

  \begin{itemize}[leftmargin=0.5cm]
    \item For thesis submissions \textbf{before August 2025}, use version \texttt{v1.1.1}.
    \item For submissions \textbf{after August 2025}, use version \texttt{v2.0.0} or later.
    \item It is recommended to always use the latest available version.
  \end{itemize}

\item To use the template on Overleaf (or locally with a LaTeX distribution), upload the entire template folder to your project workspace:

  \begin{itemize}[leftmargin=0.5cm]
    \item Create a new project on Overleaf.
    \item Click on \texttt{New Project > Upload Project} in the top-left corner.
    \item Select the \texttt{VISTEC\_Thesis\_Template.zip} file.
  \end{itemize}

\item Set up the compiler and main document settings:

  \begin{itemize}[leftmargin=0.5cm]
    \item Click the \texttt{Menu} button in the top-left corner of Overleaf.
    \item Select \texttt{XeLaTeX} from the compiler dropdown.
    \item Ensure the main document is set to \texttt{main.tex}.
    \item Make sure your TeX Live distribution is version 2023 or later.
  \end{itemize}
\end{enumerate}
\end{paragraph}




%========================
\section{Directory Structure}
\label{manual:directory-structure}
%========================

\begin{paragraph}
The template uses a folder-based structure to keep things organized. The main files and folders are:
\end{paragraph}

\begin{itemize}[leftmargin=\paritemindent]
    \item \texttt{main.tex} — The main file used to compile your thesis.
    \item \texttt{thesisinfo.tex} — Stores your title, name, advisor, program, etc.
    \item \texttt{contents/} — Contains all chapters, appendices, and special sections.
    \item \texttt{figures/} — Stores all images used in the thesis.
    \item \texttt{tables/} — (Optional) Stores separate table files.
    \item \texttt{bibliography.bib} — Contains all BibTeX references.
\end{itemize}






%========================
\section{Configuring Document Class Options}
\label{manual:class-options}
%========================

\begin{paragraph}
The \texttt{VISTEC} document class supports options to set your degree level, school, and optional layout helpers. These options must be declared in the \texttt{main.tex} file using the \verb|\documentclass| command.
\end{paragraph}

%------------------------
\subsection{Required Options}
\label{manual:class-required}
%------------------------

\begin{subparagraph}
You must provide two required options: your degree type and school/program. The valid values are shown in the table below:
\end{subparagraph}

\begin{table}[ht]
\centering
\caption{Required \texttt{documentclass} options for degree type and school/program}
\small\singlespacingplus
\begin{tabular}{ll}
\toprule
\textbf{Option} & \textbf{Description} \\
\midrule
\texttt{phd}    & Doctor of Philosophy \\
\texttt{master} & Master of Engineering \\
\midrule
\texttt{ist}    & Information Science and Technology \\
               & (Program: Information Science and Technology) \\
\texttt{ese}    & Energy Science and Engineering \\
               & (Program: Chemical Engineering) \\
\texttt{mse}    & Molecular Science and Engineering \\
               & (Program: Materials Science and Engineering) \\
\texttt{bse}    & Biomolecular Science and Engineering \\
               & (Program: Biomolecular Science and Engineering) \\
\bottomrule
\end{tabular}
\end{table}


\begin{subparagraph}
If your program is not listed, you can manually define \verb|\degreefield|, \verb|\school|, and \verb|\program| in \texttt{thesisinfo.tex}.
\end{subparagraph}

%------------------------
\subsection{Optional Layout Options}
\label{manual:class-optional}
%------------------------

\begin{subparagraph}
These optional settings help with layout debugging. Use them while editing, but disable them before final submission.
\end{subparagraph}

\begin{table}[ht]
\centering
\caption{Optional \texttt{documentclass} options for layout assistance}
\small\singlespacingplus
\begin{tabular}{ll}
\toprule
\textbf{Option} & \textbf{Purpose} \\
\midrule
\texttt{final}     & Final output (default) \\
\texttt{showframe} & Show page margins and layout boxes \\
\texttt{showgrid}  & Show grid background for positioning \\
\bottomrule
\end{tabular}
\end{table}

%------------------------
\subsection{Example Declaration}
\label{manual:class-example}
%------------------------

\begin{subparagraph}
This example sets the document class for a Ph.D. student in the ist school and enables layout frames for debugging:
\end{subparagraph}

\noindent\textbf{File: \texttt{main.tex}}\vspace{-1.5em}
\begin{minted}[breaklines, fontsize=\small]{latex}
\documentclass[phd, ist, showframe]{VISTEC}
\end{minted}







%========================
\section{Editing Thesis Metadata}
\label{manual:metadata}
%========================

\begin{paragraph}
All thesis metadata—such as the title, author name, student ID, advisor, and academic year—is defined in \texttt{thesisinfo.tex}. These values are used in the title page, approval page, and other formal sections.
\end{paragraph}

\newpage

\noindent\textbf{File: \texttt{thesisinfo.tex}}\vspace{-1.5em}
\begin{minted}[breaklines, fontsize=\small]{latex}
\title{VISTEC Thesis Template: A Complete LaTeX Thesis Preparation Version 2}
\authortitle{Mr.}
\author{Author Name}
\studentid{1888888}
\examinationdate{18}{August}{2025}
\academicyear{2025}
\advisor{Asst. Prof. Dr. Advisor Name}
\memberone{Asst. Prof. Dr. Committee Member 1}
\membertwo{Asst. Prof. Dr. Committee Member 2}
\memberthree{Asst. Prof. Dr. Committee Member 3}
\gradcommittee{Prof. Dr. Pimchai Chaiyen}
\end{minted}

%------------------------
\section{Manual Line Break in Title}
\label{manual:metadata-linebreak}
%------------------------

\begin{paragraph}
If your title is too long, it may not break naturally on the title page, approval page, or abstract page. You can insert a manual line break using \verb|\linebreak| to improve the layout. The number controls how strongly LaTeX tries to break the line.
\end{paragraph}

\begin{table}[ht]
\small\singlespacingplus
\centering
\caption{Values for \texttt{\textbackslash linebreak} and their meaning}
\begin{tabular}{cl}
\toprule
\textbf{Value} & \textbf{Effect} \\
\midrule
0 & Weak suggestion only \\
1–3 & Increasing strength of break \\
4 & Forced line break \\
\bottomrule
\end{tabular}
\end{table}

\begin{paragraph}
The example below breaks the title after the colon for better layout:
\end{paragraph}

\noindent\textbf{File: \texttt{thesisinfo.tex}}\vspace{-1.5em}
\begin{minted}[breaklines, fontsize=\small]{latex}
\title{VISTEC Thesis Template:\linebreak[2] A Complete LaTeX Thesis Preparation Version 2}
\end{minted}


%========================
\section{Organizing Chapter and Front Matter Files}
\label{manual:chapters}
%========================

\begin{paragraph}
Each part of your thesis—such as chapters, abstract, acknowledgments, and appendices—should be saved as a separate file in the \texttt{contents/} folder. These files are included in \texttt{main.tex} using the \verb|\include{}| command.
\end{paragraph}

\begin{paragraph}
The recommended structure is:
\end{paragraph}

\begin{itemize}[leftmargin=\paritemindent]
    \item \textbf{Front matter pages:}
    \begin{itemize}[leftmargin=0.5cm]
        \item \texttt{abstract.tex}
        \item \texttt{acknowledgment.tex}
        \item \texttt{abbreviations.tex}
        \item \texttt{authorbiography.tex}
    \end{itemize}
    
    \item \textbf{Main chapters:}
    \begin{itemize}[leftmargin=0.5cm]
        \item \texttt{chapter1.tex}, \texttt{chapter2.tex}, ..., \texttt{chapter5.tex}
    \end{itemize}
    
    \item \textbf{Additional sections:}
    \begin{itemize}[leftmargin=0.5cm]
        \item \texttt{appendix.tex}
    \end{itemize}
\end{itemize}

\begin{paragraph}
To include any file, use the \verb|\include{}| command in \texttt{main.tex}, like this:
\end{paragraph}

\noindent\textbf{File: \texttt{main.tex}}\vspace{-1.5em}
\begin{minted}[breaklines, fontsize=\small]{latex}
\lipsum[1] % this is dummy text, replace this line with your text

 

\keywords{Brain-computer interfaces (BCIs), Motor imagery (MI), Multi-task learning, Deep metric learning (DML), Autoencoder (AE)}


\lipsum[1][1-2] % this is dummy text, replace this line with your text


\lipsum[1][2-5] % this is dummy text, replace this line with your text


\lipsum[2] % this is dummy text, replace this line with your text
\chapter{Introduction}
\label{chapter1}





\section{Heading}


\paragraph{The chapter headings should be 14 points and any other titles should be in 12 points.  The text in the chapter body should be computer printed in 12 points Times New Roman font.}



\subsection{Sub-heading 1}

\subparagraph{
Typing should be with a spacing of 1.5 between lines, including the List of References and Appendices.
}






\subsubsection{Sub-heading 2}

\subsubparagraph{
\lipsum[1][1-3] % dummy text
}

\begin{enumerate}[itemindent=\subsubparitemindent]
\item Enumerate One
\item Enumerate Two
\item Enumerate Three
\end{enumerate}





\section{Algorithm}

\paragraph{This is an example of \autoref{chap1:algo:my-algo}.}

\begin{algorithm}[ht]
\caption{An algorithm with caption.}
\label{chap1:algo:my-algo}
\normalsize\singlespacing
\begin{algorithmic}[1] % [1] print number all lines 
    \Require $n \geq 0$
    \Ensure $y = x^n$
    \State $y \gets 1$
    \State $X \gets x$
    \State $N \gets n$
    \While{$N \neq 0$}
    \If{$N$ is even}
        \State $X \gets X \times X$
        \State $N \gets \frac{N}{2}$  \Comment{This is a comment}
    \ElsIf{$N$ is odd}
        \State $y \gets y \times X$
        \State $N \gets N - 1$
    \EndIf
    \EndWhile
\end{algorithmic}
\end{algorithm}



\section{Equation} 

\paragraph{As an illustration of \LaTeX's mathematics formatting,
\autoref{chap1:eq:renyi} is the definition of {\em R\'enyi entropy} and \autoref{chap1:eq:total-loss} is the total loss function:
}

%%%%%%%%%%%%%%%%%%%%%%%%% EQUATION  %%%%%%%%%%%%%%%%%%%%%%%%% 
\begin{equation}
\label{chap1:eq:renyi}
H_{\alpha}(X) =
\frac{1}{1-\alpha}
\log \left(\sum_{x \in {\cal X}}P[X=x]^{\alpha} \right) .
\end{equation}

%%%%%%%%%%%%%%%%%%%%%%%%%  EQUATION %%%%%%%%%%%%%%%%%%%%%%%%% 
\begin{equation} 
\label{chap1:eq:total-loss}
\begin{aligned}
\mathcal{L}_{\textrm{total}} = \frac{1}{N}\sum_{i=1}^{N}\{w_i\mathcal{L}_i\}. 
\end{aligned}
\end{equation}


\paragraph{
\lipsum[1][1-3] % dummy text
}
...
%----------------------------------------
\appendix
\chapter{Proofs for Chapter 3}
\label{appendix}
%----------------------------------------

%========================
\section{Proof of Lemma}
%========================
\begin{paragraph}
This section provides the detailed proof of the lemma stated in Chapter 3. The proof follows standard steps in mathematical derivation and demonstrates the validity of the stated result.
\end{paragraph}

\end{minted}







%========================
\section{Structuring Headings and References}
\label{manual:headings}
%========================

\begin{paragraph}
To keep your document well-organized, use headings consistently: \verb|\section|, \verb|\subsection|, \verb|\subsubsection|. Add \verb|\label| after each heading to create a reference target. Use \verb|\autoref| to reference them automatically with the correct prefix (e.g., ``Section'').
\end{paragraph}

\noindent\textbf{File: \texttt{contents/chapter1.tex}}\vspace{-1.5em}
\begin{minted}[fontsize=\small, breaklines]{latex}
\section{Introduction}
\label{sec:intro}

\begin{paragraph}
This is a paragraph. Refer to \autoref{sec:background}.
\end{paragraph}

\subsection{Background}
\label{sec:background}

\begin{subparagraph}
This is a subparagraph that expands on background context.
\end{subparagraph}

\subsubsection{Detailed Context}
\label{subsec:detail}

\begin{subsubparagraph}
This subsubparagraph elaborates on the content in \autoref{sec:background}.
\end{subsubparagraph}
\end{minted}

\newpage

\textbf{Output:}\vspace{0.5em}

\begin{tcolorbox}[colback=white, colframe=black, sharp corners, boxrule=0.4pt]

\textbf{1\hspace{1.1cm}Introduction}\vspace{6pt}

\hspace{1.25cm}This is a paragraph. Refer to Section 1.2.\vspace{6pt}

\hspace{1.25cm}\textbf{1.2 Background}\vspace{6pt}

\hspace{1.95cm}This is a subparagraph that expands on background context.\vspace{6pt}

\hspace{1.95cm}\textbf{1.2.1 Detailed Context}\vspace{6pt}

\hspace{2.95cm}This subsubparagraph elaborates on the content in Section 1.2.

\end{tcolorbox}

%------------------------
\subsection{Referencing Tables, Figures, and Equations}
\label{manual:ref-table-figure}
%------------------------

\begin{subparagraph}
To reference tables, figures, or equations, use \verb|\label| and \verb|\autoref|. Always place the \verb|\label| right after the \verb|\caption| or at the end of the equation environment. This ensures correct automatic prefixing like ``Table'', ``Figure'', or ``Equation''.
\end{subparagraph}

\noindent\textbf{File: \texttt{contents/chapter1.tex}}\vspace{-1.5em}
\begin{minted}[fontsize=\small, breaklines]{latex}
% Referencing a table, figure, and equation
As shown in \autoref{tab:summary}, \autoref{fig:sample}, and \autoref{eq:loss}, our results are consistent.

% Table example
\begin{table}[ht]
\small\singlespacingplus
\centering
\caption{Summary of accuracy across datasets.}
\label{tab:summary}
    \begin{tabular}{lll}
        \toprule
        Dataset & Subjects & Accuracy \\
        \midrule
        A & 10 & 85.2\% \\
        B & 12 & 88.6\% \\
        \bottomrule
    \end{tabular}
\end{table}

% Figure example
\begin{figure}[ht]
    \centering
    \includegraphics[width=0.9\linewidth]{figures/sample_plot.pdf}
    \caption{Accuracy comparison between models.}
    \label{fig:sample}
\end{figure}

% Equation (not shown in output box)
\begin{equation}
\mathcal{L}_{\text{total}} = \sum_{t=1}^{T} \alpha_t \cdot \mathcal{L}_t
\label{eq:loss}
\end{equation}
\end{minted}

\textbf{Output:}\vspace{0.5em}

\begin{tcolorbox}[colback=white, colframe=black, sharp corners, boxrule=0.4pt]
\hspace{1.25cm}As shown in Table 1, Figure 1, and \autoref{eq:loss}, our results are consistent.

\vspace{6pt}
\textbf{Table 1} Summary of accuracy across datasets.

\vspace{-6pt}
\begin{center}
[SAMPLE TABLE]
\end{center}

\begin{center}
[SAMPLE PLOT]
\end{center}
\vspace{-6pt}
\textbf{Figure 1} Accuracy comparison between models.

\begin{equation}
\mathcal{L}_{\text{total}} = \sum_{t=1}^{T} \alpha_t \cdot \mathcal{L}_t
\label{eq:loss}
\end{equation}

\end{tcolorbox}




%========================
\section{Customizing List Indentation}
\label{manual:indentation}
%========================

\begin{paragraph}
List indentation improves readability by visually separating content by level. This template provides three predefined indentation lengths:
\end{paragraph}

\begin{table}[ht]
\small\singlespacingplus
\centering
\caption{Predefined macros for list indentation}
\begin{tabular}{ll}
\toprule
\textbf{Macro} & \textbf{Indent Size} \\
\midrule
\verb|\paritemindent| & 1.65cm — First-level lists (main paragraph level) \\
\verb|\subparitemindent| & 2.8cm — Second-level lists (nested or subparagraph level) \\
\verb|\subsubparitemindent| & 4cm — Third-level lists (deeply nested content) \\
\bottomrule
\end{tabular}
\end{table}

\begin{paragraph}
Below is an example of how to apply these indentation macros in \texttt{enumerate} and \texttt{itemize} environments. You can also use specific units like \texttt{cm} or \texttt{pt} when more control is needed.
\end{paragraph}
\newpage

\noindent\textbf{File: \texttt{contents/xxx.tex}}\vspace{-1.5em}
\begin{minted}[fontsize=\small, breaklines]{latex}
% Custom indentation using predefined macros
\begin{enumerate}[itemindent=\paritemindent]
  \item First-level list item (using paritemindent)
\end{enumerate}

\begin{enumerate}[itemindent=\subparitemindent]
  \item Second-level list item (using subparitemindent)
\end{enumerate}

\begin{enumerate}[itemindent=\subsubparitemindent]
  \item Third-level list item (using subsubparitemindent)
\end{enumerate}

% Manual indentation using fixed units
\begin{itemize}[itemindent=2cm]
  \item Manually indented item using 2cm
\end{itemize}

\end{minted}

%--- Visual Output Box ---
\textbf{Output:}\vspace{0.5em}

\begin{tcolorbox}[colback=white, colframe=black, sharp corners, boxrule=0.4pt]
% Custom indentation using predefined macros
\begin{enumerate}[itemindent=\paritemindent]
  \item First-level list item (using paritemindent)
\end{enumerate}

\begin{enumerate}[itemindent=\subparitemindent]
  \item Second-level list item (using subparitemindent)
\end{enumerate}

\begin{enumerate}[itemindent=\subsubparitemindent]
  \item Third-level list item (using subsubparitemindent)
\end{enumerate}

% Manual indentation using fixed units
\begin{itemize}[itemindent=2cm]
  \item Manually indented item using 2cm
\end{itemize}

\end{tcolorbox}





%========================
\section{Font Size}
\label{manual:font-size}
%========================

\begin{paragraph}
This template customizes the default font settings for improved readability. The default font size is \textbf{12pt}. You may override it using any of the commands below.
\end{paragraph}

\begin{table}[h]
\centering
\caption{Font size commands with visual examples}
\renewcommand{\arraystretch}{1.2}
\begin{tabular}{ll>{\raggedright\arraybackslash}p{5cm}}
\toprule
\textbf{Command} & \textbf{Font Size (pt)} & \textbf{Example Text} \\
\midrule
\verb|\HUGE|         & 24pt & {\HUGE Some text} \\
\verb|\huge|         & 20pt & {\huge Some text} \\
\verb|\LARGE|        & 18pt & {\LARGE Some text} \\
\verb|\Large|        & 16pt & {\Large Some text} \\
\verb|\large|        & 14pt & {\large Some text} \\
\verb|\normalsize|   & 12pt (default) & Some text \\
\verb|\small|        & 11pt & {\small Some text} \\
\verb|\footnotesize| & 10pt & {\footnotesize Some text} \\
\verb|\scriptsize|   & 9pt  & {\scriptsize Some text} \\
\verb|\tiny|         & 8pt  & {\tiny Some text} \\
\bottomrule
\end{tabular}
\end{table}

\newpage

\noindent\textbf{File: \texttt{contents/xxx.tex}}\vspace{-1.5em}
\begin{minted}[fontsize=\small]{latex}
{\Large This should appear larger.}

{\small This should appear smaller.}
\end{minted}

\textbf{Output:}\vspace{0.5em}

\begin{tcolorbox}[colback=white, colframe=black, sharp corners, boxrule=0.4pt]
{\Large This should appear larger.}

{\small This should appear smaller.}
\end{tcolorbox}





%========================
\section{Formatting Tips and Layout Troubleshooting}
%========================

\begin{paragraph}
This section provides helpful solutions to common formatting issues in your thesis, such as overfull lines, missing continuation headers, and manual page breaks.
\end{paragraph}

%------------------------
\subsection{Fixing Overfull \texttt{\textbackslash hbox} Warnings}
\label{manual:overflow-fix}
%------------------------

\begin{subparagraph}
An \textcolor{orange}{``Overfull \texttt{\textbackslash hbox}''} warning occurs when LaTeX cannot break a long word or line within the page margins. There are two typical solutions:
\end{subparagraph}

\begin{itemize}[leftmargin=\subparitemindent]
  \item Use \verb|\hyphenation{}| in the preamble to define custom hyphenation points for specific words.
  \item Insert a manual line break using \verb|\break| or \verb|\newline| in the document body. (It is recommended to use \verb|\break| for better formatting control.)
\end{itemize}


%------------------------
\begin{subparagraph}
\textbf{Example 1: Using Hyphenation Rules \textcolor{red}{(recommended for Title)}}\\

Place these commands in the preamble to help LaTeX break long words:
\end{subparagraph}

\noindent\textbf{File: \texttt{main.tex}}\vspace{-1.5em}
\begin{minted}[breaklines, fontsize=\small]{latex}
\hyphenation{neurorehabi-litation} % Breaks as neurorehabi-litation
\hyphenation{multi-modal}          % Breaks as multi-modal
\hyphenation{inherent}             % Do not hyphenate this word
\end{minted}


%------------------------
\begin{subparagraph}
\textbf{Example 2: Manual Line Break \textcolor{red}{(recommended for body text)}}\\

Insert \verb|\break| and \verb|\newline| at the desired point in a long sentence:
\end{subparagraph}

\noindent\textbf{File: \texttt{contents/xxx.tex}}\vspace{-1.5em}
\begin{minted}[breaklines, fontsize=\small]{latex}
This sentence is too long and exceeds the margin, so we insert a break here with proper\break
indentation on the next line.

This sentence is too long and exceeds the margin, so we insert a newline here with proper\newline
indentation on the next line.
\end{minted}

\textbf{Output:}\vspace{0.5em}

\begin{tcolorbox}[colback=white, colframe=black, sharp corners, boxrule=0.4pt]
This sentence is too long and exceeds the margin, so we insert a break here with proper\break
indentation on the next line.

This sentence is too long and exceeds the margin, so we insert a newline here with proper\newline
indentation on the next line.
\end{tcolorbox}

%------------------------
\subsection{Forcing a Page Break}
%------------------------

\begin{subparagraph}
To manually start a new page, use:
\end{subparagraph}

\begin{minted}[breaklines, fontsize=\small]{latex}
\newpage
\end{minted}

%------------------------
\subsection{Fixing Missing Continuation Headers in Lists}
%------------------------


\begin{subparagraph}
If a continuation header (e.g., \texttt{(Cont.)}) does not appear on the second page of a list, insert a dummy entry to trigger it. These entries are invisible but ensure correct layout. Uncomment the relevant lines based on the list affected.
\end{subparagraph}

\begin{subparagraph}
Add this at the end of your \texttt{main.tex}:
\end{subparagraph}

\noindent\textbf{File: \texttt{main.tex}}\vspace{-1.5em}
\begin{minted}[breaklines, fontsize=\small]{latex}
\addtocontents{lot}{\protect\contentsline{table}{\phantom{Dummy Invisible Table Entry}}{\phantom{\thepage}}{}}
\addtocontents{lof}{\protect\contentsline{figure}{\phantom{Dummy Invisible Figure Entry}}{\phantom{\thepage}}{}}
% \addtocontents{toc}{\protect\contentsline{chapter}{\phantom{Dummy Invisible ToC Entry}}{\phantom{\thepage}}{}}
\end{minted}

%------------------------
\subsection{Forcing Continuation Headers in the List of Abbreviations}
%------------------------

\begin{subparagraph}
If the continuation header in the List of Abbreviations does not appear automatically, use \verb|\newpage| to manually break the page.
\end{subparagraph}

\begin{subparagraph}
Example:
\end{subparagraph}

\noindent\textbf{File: \texttt{contents/abbreviations.tex}}\vspace{-1.5em}
\begin{minted}[breaklines, fontsize=\small]{latex}
\newabbr{EEG}{Electroencephalogram}
\newabbr{MI}{Motor Imagery}
\newabbr{CNN}{Convolutional Neural Network}
\newabbr{\ce{H2O}}{Water}
\newpage % Force second page
\newabbr{DBU}{1,8-diazabicyclo[5.4.0]-7-undecene}
\end{minted}


\end{document} 