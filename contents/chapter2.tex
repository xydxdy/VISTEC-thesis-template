%----------------------------------------
\chapter{How to Use This Template}
\label{chapter2}
%----------------------------------------

%========================
\section{Overview}
%========================
\begin{paragraph}
This chapter provides guidance on how to effectively use and customize the VISTEC \LaTeX{} thesis template. It explains the general structure, key files, and recommended practices to help users quickly adapt the template for their own thesis writing.
\end{paragraph}

%========================
\section{Directory Structure}
%========================
\begin{paragraph}
The template is organized into clearly separated folders and files to simplify management:

\begin{itemize}[leftmargin=\paritemindent]
    \item \texttt{main.tex} — The main file to compile your thesis.
    \item \texttt{thesisinfo.tex} — Define your title, author information, advisor, and committee.
    \item \texttt{contents/} — Contains all chapter, appendix, and special section files.
    \item \texttt{figures/} — Store all figures, images, and plots used in the thesis.
    \item \texttt{tables/} — Store external table files if needed.
    \item \texttt{bibliography.bib} — Your BibTeX bibliography database.
\end{itemize}
\end{paragraph}

%========================
\section{Editing Thesis Metadata}
%========================
\begin{paragraph}
Edit \texttt{thesisinfo.tex} to set your thesis title, author name, student ID, academic year, advisor, committee members, and program information. These metadata fields automatically populate the title page, approval page, and other formal sections.
\end{paragraph}

%========================
\section{Adding Content to Chapters}
%========================
\begin{paragraph}
Each main chapter (e.g., Introduction, Background, Investigations, Conclusion) should be placed under \texttt{contents/} and included using \verb|\include{}| in \texttt{main.tex}. You can create additional chapter files following the provided structure, and organize sections, figures, tables, algorithms, and citations inside them.
\end{paragraph}

%========================
\section{Inserting Figures, Tables, and Equations}
%========================
\begin{paragraph}
Use standard LaTeX environments to insert figures, tables, and equations. Examples are provided in \texttt{contents/chapter2.tex}. Figures and tables should include proper captions, labels, and cross-references using \verb|\autoref{}| to maintain consistency.
\end{paragraph}

%========================
\section{Citing References}
%========================
\begin{paragraph}
Manage your references in \texttt{bibliography.bib} using BibTeX format. Cite references in your chapters using \verb|\cite{}| for in-text citations and \verb|\fullcite{}| when listing complete citations. The Vancouver bibliography style is automatically applied.
\end{paragraph}

%========================
\section{How to Use \LaTeX}
%========================
\begin{paragraph}
If you are new to \LaTeX, it is recommended to start with basic tutorials to understand fundamental concepts such as document structure, commands, environments, and referencing. A good starting point is the Overleaf online guide available at:

\begin{center}
\href{https://www.overleaf.com/learn}{\texttt{https://www.overleaf.com/learn}}
\end{center}

The Overleaf Learn platform provides comprehensive, beginner-friendly resources covering topics from basic document setup to advanced formatting and bibliography management. Familiarity with these concepts will significantly improve your ability to customize and work efficiently with this thesis template.
\end{paragraph}

%========================
\section{Compiling the Thesis}
%========================
\begin{paragraph}
Use \texttt{pdfLaTeX} as the compiler. A typical compilation sequence includes:
\begin{itemize}[leftmargin=\paritemindent]
    \item First, run \texttt{pdflatex main.tex} to generate auxiliary files.
    \item Then, run \texttt{bibtex main} to generate the bibliography.
    \item Finally, run \texttt{pdflatex main.tex} twice to resolve cross-references.
\end{itemize}

Alternatively, tools like \texttt{latexmk} or IDEs like Overleaf, TeXShop, and VS Code with LaTeX Workshop can automate this process.
\end{paragraph}

%========================
\section{Notes and Recommendations}
%========================
\begin{paragraph}
\begin{itemize}[leftmargin=\paritemindent]
    \item Regularly back up your files.
    \item Keep figures and tables in separate folders to maintain a clean project.
    \item Use consistent citation and figure labeling conventions.
    \item Compile often to catch errors early.
\end{itemize}
\end{paragraph}
