%----------------------------------------
\chapter{Introduction}
\label{chapter1}
%----------------------------------------

%========================
\section{Motivation}
\label{ch1:sec:motivation}
%========================
\begin{paragraph}
Brain-Computer Interfaces (BCIs) have emerged as a transformative technology enabling direct communication between the human brain and external devices. This field holds immense potential for applications in assistive technologies, neurorehabilitation, and human-computer interaction, offering new hope for individuals with motor disabilities. However, designing effective BCIs remains challenging due to the inherent variability in brain signals, the presence of noise, and the limited availability of high-quality datasets.

Despite significant advancements in machine learning and signal processing, many current BCI systems struggle with generalization across users, sessions, and tasks. Addressing these challenges requires innovative approaches to improve robustness, adaptability, and scalability. This thesis is motivated by the need to develop methodologies that not only enhance the performance of EEG-based BCIs but also make them more reliable and practical for real-world deployment.
\end{paragraph}


%========================
\section{Contributions}
\label{ch1:sec:Contributions}
%========================
\begin{paragraph}
This thesis makes the following key contributions:
\end{paragraph}

\begin{itemize}[leftmargin=\paritemindent]
    \item We introduce a novel experimental paradigm that addresses key limitations in the current research.
    \item We propose a novel algorithm that enhances learning performance across multiple tasks.
\end{itemize}

%========================
\section{Outline}
\label{ch1:sec:outline}
%========================
\begin{paragraph}
This thesis is organized into the following chapters:

\textbf{\autoref{chapter1} Introduction:}  
Introduces the research motivation, key contributions, and provides an overview of the thesis structure.

\textbf{\autoref{chapter2} Background:}  
Provides a comprehensive overview of the fundamental concepts, theoretical foundations, and prior research that form the basis of this thesis.

\textbf{\autoref{chapter3} How to Use This Template:}  
Offers practical guidance on using the VISTEC {\LaTeX} thesis template, along with examples demonstrating how to format and organize paragraphs, sections, equations, algorithms, tables, figures, citations, and footnotes.

\textbf{\autoref{chapter4} VISTEC Thesis Formatting:}  
Illustrates standardized formatting examples for a VISTEC thesis, covering headings, equations, algorithms, tables, figures, citations, and footnotes to ensure consistency throughout the document.

\textbf{\autoref{chapter5} Conclusion:}  
Summarizes the major findings, discusses their implications, and suggests future research directions.

\textbf{\autoref{appendix} Proofs Supporting Investigation:}  
Presents supplementary materials, including detailed proofs, additional results, and extended discussions that support the main chapters.
\end{paragraph}
