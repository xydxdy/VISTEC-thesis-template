%----------------------------------------
\chapter{Template Usage Examples}
\label{chapter3}
%----------------------------------------

%========================
\section{Heading}
%========================
\begin{paragraph}
This is a paragraph under the main section. It introduces the overall content of the section in a general manner.
\end{paragraph}

%========================
\subsection{Sub-heading}
%========================
\begin{subparagraph}
This is a subparagraph under the first subsection. It provides additional detail or clarification related to the subsection's topic.
\end{subparagraph}

%========================
\subsubsection{Second-level Sub-heading}
%========================
\begin{subsubparagraph}
This is a subsubparagraph under the second-level subheading. It is typically used for listing or elaborating fine-grained points.
\end{subsubparagraph}

\begin{enumerate}[itemindent=\subsubparitemindent]
    \item This is the first item in the enumerated list.
    \item This is the second item in the enumerated list.
    \item This is the third item in the enumerated list.
\end{enumerate}

%========================
\section{Equation}
%========================
\begin{paragraph}
As an illustration of \LaTeX's mathematics formatting, \autoref{ch3:eq:renyi} is the definition of {\em R\'enyi entropy} and \autoref{ch3:eq:total-loss} is the total loss function:
\end{paragraph}

%----- Equation: Renyi Entropy -----
\begin{equation}
\label{ch3:eq:renyi}
H_{\alpha}(X) =
\frac{1}{1-\alpha}
\log \left(\sum_{x \in {\cal X}}P[X=x]^{\alpha} \right) .
\end{equation}

%----- Equation: Total Loss -----
\begin{equation}
\label{ch3:eq:total-loss}
\begin{aligned}
\mathcal{L}_{\textrm{total}} = \frac{1}{N}\sum_{i=1}^{N}\{w_i\mathcal{L}_i\}.
\end{aligned}
\end{equation}

%========================
\section{Algorithm}
%========================
\begin{paragraph}
This is an example of \autoref{ch3:algo:my-algo}.
\end{paragraph}

\begin{algorithm}[h]
\caption{An algorithm with caption.}
\label{ch3:algo:my-algo}
\normalsize\singlespacing
\begin{algorithmic}[1] % [1] print number on all lines
    \Require $n \geq 0$
    \Ensure $y = x^n$
    \State $y \gets 1$
    \State $X \gets x$
    \State $N \gets n$
    \While{$N \neq 0$}
        \If{$N$ is even}
            \State $X \gets X \times X$
            \State $N \gets \frac{N}{2}$ \Comment{This is a comment}
        \ElsIf{$N$ is odd}
            \State $y \gets y \times X$
            \State $N \gets N - 1$
        \EndIf
    \EndWhile
\end{algorithmic}
\end{algorithm}

%========================
\section{Table}
%========================
\begin{paragraph}
\LaTeX table generators, such as \href{https://www.tablesgenerator.com/}{TablesGenerator.com}\footnotemark{}, can help you easily create well-formatted tables\footnotetext{https://www.tablesgenerator.com/}.
Here, \autoref{ch3:table:results} is an example of a table generated using the tool.
\end{paragraph}

%----- Table: Classification Results -----
\begin{table}[ht]
\caption{Classification performance. An asterisk ($^*$) indicates values that are significantly different from the others ($p<0.05$).}
\label{ch3:table:results}
\centering
\normalsize\singlespacingplus
    \begin{tabular}{@{}ccc@{}}
    \toprule
    \multirow{2}{*}{\textbf{Comparison Model}} & \multicolumn{2}{c}{\textbf{Subject-independent}}       \\ \cmidrule(l){2-3} 
                                               & \textbf{Accuracy $\pm$ SD}          & \textbf{F1-score $\pm$ SD}         \\ \midrule
    FBCSP-SVM                                  & $64.96 \pm 12.70$          & $65.25 \pm 15.14$         \\
    Deep Convnet                               & $68.33 \pm 15.33$          & $70.20 \pm 15.18$         \\
    EEGNet-8,2                                 & $68.84 \pm 13.87$          & $70.39 \pm 14.30$         \\
    Spectral-Spatial CNN                       & $68.27 \pm 13.56$          & $65.86 \pm 17.37$         \\
    MIN2Net                                    & $\mathbf{72.03 \pm 14.04^*}$ & $\mathbf{72.62 \pm 14.14^*}$ \\ \bottomrule
    \end{tabular}%
\end{table}

%========================
\section{Figure}
%========================
\begin{paragraph}
Here are examples of figures in a thesis: \autoref{ch3:fig:fig-A} illustrates a standard image inclusion, while \autoref{ch3:fig:mychemfig} shows a chemical reaction diagram generated with \texttt{chemfig}.
\end{paragraph}

%----- Figure: Standard Image -----
\begin{figure}[ht]
    \centering
    \includegraphics[width=1\columnwidth]{figures/ch3/A.pdf} % Could be .png or .jpg
    \longcaption[This is a long figure caption example for an image.]{This figure demonstrates how to include a standard image (e.g., PDF, PNG, JPG) into your document. Captions longer than one line should be aligned left and indented after the first line.}
    \label{ch3:fig:fig-A}
\end{figure}

%----- Figure: Chemfig Reaction -----
\begin{figure}[ht]
    \centering
    \chemnameinit{\chemfig{R-C(-[:-30]OH)=[:30]O}} % Optional molecule at the top
    \schemestart
        \chemname{\chemfig{R’OH}}{Alcohol}
        \+
        \chemname{\chemfig{R-C(-[:-30]OH)=[:30]O}}{Carboxylic acid}
        \arrow(.mid east--.mid west)
        \chemname{\chemfig{R-C(-[:-30]OR’)=[:30]O}}{Ester}
        \+
        \chemname{\chemfig{H_2O}}{Water}
    \schemestop
    \chemnameinit{}
    \caption{An esterification reaction illustrated using the \texttt{chemfig} package.}
    \label{ch3:fig:mychemfig}
\end{figure}

%========================
\section{Citation}
%========================
\begin{paragraph}
This is an example of how to cite previous work, such as \cite{min2net}, or multiple sources like \cite{hu79, somework2020, tonio_paper}. Ensure that the corresponding BibTeX entries are added to the \texttt{bibliography.bib} file before citing.

Below is an example BibTeX entry:
\begin{verbatim}
    @ARTICLE{dummy2022example,
      author  = {Doe, John and Smith, Jane and Roe, Richard},
      journal = {Journal of Example Studies}, 
      title   = {A Dummy Title for Demonstration Purposes}, 
      year    = {2022},
      volume  = {42},
      number  = {1},
      pages   = {1--10}
    }
\end{verbatim}
\end{paragraph}

%========================
\section{Footnote}
%========================
\begin{paragraph}
I'm writing to test the \verb|\footnotemark| and \verb|\footnotetext| commands. 
You can insert a footnote marker using the \verb|\footnotemark|\footnotemark{} command and later typeset the footnote text by writing 
\verb|\footnotetext{Example footnote.}|. \footnotetext{Example footnote.}

Let's do one more to see the result\footnotemark{}, which I'll comment on within the footnote.\footnotetext{Specifically, I'd write comments in this one.}
\end{paragraph}
