\chapter{Background}		
\label{chapter2}


\section{Heading}

\begin{paragraph}
The chapter headings should be 14 points and any other titles should be in 12 points.  The text in the chapter body should be computer printed in 12 points Times New Roman font.
\end{paragraph}

\subsection{Sub-heading 1}

\begin{subparagraph}
Typing should be with a spacing of 1.5 between lines, including the List of References and Appendices.
\end{subparagraph}






\subsubsection{Sub-heading 2}

\begin{subsubparagraph}
Subheading 2 provides an example of items presented in an enumerated list.
\end{subsubparagraph}

\begin{enumerate}[itemindent=\subsubparitemindent]
\item Enumerate One
\item Enumerate Two
\item Enumerate Three
\end{enumerate}


\section{Equation} 

\begin{paragraph}
As an illustration of \LaTeX's mathematics formatting,
\autoref{ch2:eq:renyi} is the definition of {\em R\'enyi entropy} and \autoref{ch2:eq:total-loss} is the total loss function:
\end{paragraph}

%%%%%%%%%%%%%%%%%%%%%%%%% EQUATION  %%%%%%%%%%%%%%%%%%%%%%%%% 
\begin{equation}
\label{ch2:eq:renyi}
H_{\alpha}(X) =
\frac{1}{1-\alpha}
\log \left(\sum_{x \in {\cal X}}P[X=x]^{\alpha} \right) .
\end{equation}

%%%%%%%%%%%%%%%%%%%%%%%%%  EQUATION %%%%%%%%%%%%%%%%%%%%%%%%% 
\begin{equation} 
\label{ch2:eq:total-loss}
\begin{aligned}
\mathcal{L}_{\textrm{total}} = \frac{1}{N}\sum_{i=1}^{N}\{w_i\mathcal{L}_i\}. 
\end{aligned}
\end{equation}


\section{Algorithm}

\begin{paragraph}
This is an example of \autoref{ch2:algo:my-algo}.
\end{paragraph}

\begin{algorithm}[h]
\caption{An algorithm with caption.}
\label{ch2:algo:my-algo}
\normalsize\singlespacing
\begin{algorithmic}[1] % [1] print number all lines 
    \Require $n \geq 0$
    \Ensure $y = x^n$
    \State $y \gets 1$
    \State $X \gets x$
    \State $N \gets n$
    \While{$N \neq 0$}
    \If{$N$ is even}
        \State $X \gets X \times X$
        \State $N \gets \frac{N}{2}$  \Comment{This is a comment}
    \ElsIf{$N$ is odd}
        \State $y \gets y \times X$
        \State $N \gets N - 1$
    \EndIf
    \EndWhile
\end{algorithmic}
\end{algorithm}



\section{Table}

\begin{paragraph}
LaTeX table generators, such as \href{https://www.tablesgenerator.com/}{TablesGenerator.com}\footnotemark{}, can help you easily create well-formatted tables\footnotetext{https://www.tablesgenerator.com/}. 
Here, \autoref{ch2:table:results} is an example of a table generated using the tool.
\end{paragraph}
    

%%%%%%%%%%%%%%%%%%%%%%%%% TABLE BEGIN %%%%%%%%%%%%%%%%%%%%%%%%% 
%%%%% Create LaTeX table from https://www.tablesgenerator.com/
\begin{table}[ht]
\caption{Classification performance. An asterisk ($^*$) indicates values that are significantly different from the others ($p<0.05$).}
\label{ch2:table:results}
\centering
\normalsize\singlespacingplus
    \begin{tabular}{@{}ccc@{}}
    \toprule
    \multirow{2}{*}{\textbf{Comparison Model}} & \multicolumn{2}{c}{\textbf{Subject-independent}}       \\ \cmidrule(l){2-3} 
                                               & \textbf{Accuracy $\pm$ SD}          & \textbf{F1-score $\pm$ SD}         \\ \midrule
    FBCSP-SVM                                  & $64.96 \pm 12.70$          & $65.25 \pm 15.14$         \\
    Deep Convnet                               & $68.33 \pm 15.33$          & $70.20 \pm 15.18$         \\
    EEGNet-8,2                                 & $68.84 \pm 13.87$          & $70.39 \pm 14.30$         \\
    Spectral-Spatial CNN                       & $68.27 \pm 13.56$          & $65.86 \pm 17.37$         \\
    MIN2Net                                    & $\mathbf{72.03 \pm 14.04^*}$ & $\mathbf{72.62 \pm 14.14^*}$ \\ \bottomrule
    \end{tabular}%
\end{table}
%%%%%%%%%%%%%%%%%%%%%%%%% TABLE END %%%%%%%%%%%%%%%%%%%%%%%%%  





\section{Figure}

\begin{paragraph}
Here are the example of Figure \ref{ch2:fig:fig-A} and Figure \ref{ch2:fig:mychemfig}.
\end{paragraph}


%%%%%%%%%%%%%%%%%%%%%%%%% FIGURE BEGIN %%%%%%%%%%%%%%%%%%%%%%%%% 
\begin{figure}[ht]
\centering
\includegraphics[width=1\columnwidth]{figures/ch2/A.pdf}
\longcaption[This is the example of very long caption of images.]{For the figure caption which contains more than 1 line, it should align left throughout the thesis. The second and other lines need to be aligned with the first letter of the first line.}
\label{ch2:fig:fig-A}
\end{figure}
%%%%%%%%%%%%%%%%%%%%%%%%% FIGURE END %%%%%%%%%%%%%%%%%%%%%%%%% 


%%%%%%%%%%%%%%%%%%%%%%%%% FIGURE BEGIN %%%%%%%%%%%%%%%%%%%%%%%%% 
\begin{figure}[ht]
    \centering
    % From the documentation
    % https://ctan.mirror.garr.it/mirrors/ctan/macros/generic/chemfig/chemfig-en.pdf
    \chemnameinit{\chemfig{R-C(-[:-30]OH)=[:30]O}}
    \schemestart
    \chemname{\chemfig{R’OH}}{Alcohol}
    \+
    \chemname{\chemfig{R-C(-[:-30]OH)=[:30]O}}{Carboxylic acid}
    \arrow(.mid east--.mid west)
    \chemname{\chemfig{R-C(-[:-30]OR’)=[:30]O}}{Ester}
    \+
    \chemname{\chemfig{H_2O}}{Water}
    \schemestop
    \chemnameinit{}
    \caption{The caption on this figure.}
    \label{ch2:fig:mychemfig}
\end{figure}


\section{Citation}
\begin{paragraph}
This is an example of how to cite previous work, such as \cite{min2net}, or multiple sources like \cite{hu79, somework2020, tonio_paper}. Ensure that the corresponding BibTeX entries are added to the \texttt{bibliography.bib} file before citing.

Below is an example BibTeX entry:

\begin{verbatim}
    @ARTICLE{dummy2022example,
      author  = {Doe, John and Smith, Jane and Roe, Richard},
      journal = {Journal of Example Studies}, 
      title   = {A Dummy Title for Demonstration Purposes}, 
      year    = {2022},
      volume  = {42},
      number  = {1},
      pages   = {1--10}
    }
    \end{verbatim}

\end{paragraph}



\section{Footnote}

\begin{paragraph}
I'm writing to test the \verb|\footnotemark| and \verb|\footnotetext| commands. 
You can insert a footnote marker using the \verb|\footnotemark|\footnotemark{}
command and later, when you're ready, typeset the footnote text by writing 
\verb|\footnotetext{Here's the footnote.}|. \footnotetext{Here's the footnote.}

Let's do one more to see the result\footnotemark{} which I'll comment on within the
footnote\footnotetext{Specifically, I'd write comments in this one.}.
\end{paragraph}