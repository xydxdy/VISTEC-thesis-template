%----------------------------------------
\chapter{How to Use This Template}
\label{chapter3}
%----------------------------------------

%========================
\section{Overview}
%========================
\begin{paragraph}
This chapter provides guidance on how to effectively use and customize the VISTEC {\LaTeX} thesis template. It explains the general structure, key files, recommended practices, and demonstrates common {\LaTeX} formatting examples such as inserting figures, tables, equations, algorithms, and citations.
\end{paragraph}

%========================
\section{Directory Structure}
%========================
\begin{paragraph}
The template is organized into clearly separated folders and files to simplify management:

\begin{itemize}[leftmargin=\paritemindent]
    \item \texttt{main.tex} — The main file to compile your thesis.
    \item \texttt{thesisinfo.tex} — Define your title, author information, advisor, and committee.
    \item \texttt{contents/} — Contains all chapter, appendix, and special section files.
    \item \texttt{figures/} — Store all figures, images, and plots used in the thesis.
    \item \texttt{tables/} — Store external table files if needed.
    \item \texttt{bibliography.bib} — Your BibTeX bibliography database.
\end{itemize}
\end{paragraph}

%========================
\section{Setting Up Document Class Options}
%========================
\begin{paragraph}
The \texttt{VISTEC} document class supports several options to customize the thesis according to the degree type, school, and program. Proper configuration of these options ensures that the generated thesis document meets the official formatting standards. 

\textbf{The document class options must be specified in the \texttt{main.tex} file.}
\end{paragraph}

%========================
\subsection{Required Options}
%========================
\begin{subparagraph}
Two required options must be specified when declaring the document class in \texttt{main.tex}:

\begin{itemize}[leftmargin=\subparitemindent]
    \item \textbf{Degree Type:}
    \begin{itemize}[leftmargin=1em]
        \item \texttt{phd} — for \verb|\degreefield{Doctor of Philosophy}|
        \item \texttt{master} — for \verb|\degreefield{Master of Engineering}|
    \end{itemize}

    \item \textbf{School and Program:}
    \begin{itemize}[leftmargin=1em]
        \item \texttt{ist} — School of Information Science and Technology \\ \quad\quad (Program: Information Science and Technology)
        \item \texttt{mse} — School of Molecular Science and Engineering \\ \quad\quad (Program: Materials Science and Engineering)
        \item \texttt{ese} — School of Energy Science and Engineering \\ \quad\quad (Program: Chemical Engineering)
        \item \texttt{bse} — School of Biomolecular Science and Engineering \\ \quad\quad (Program: Biomolecular Science and Engineering)
    \end{itemize}
\end{itemize}

If your program is not among the predefined options, you must manually specify the \verb|\degreefield|, \verb|\school|, and \verb|\program| fields in \texttt{thesisinfo.tex}.
\end{subparagraph}

%========================
\subsection{Optional Options}
%========================
\begin{subparagraph}
The document class also provides optional settings that control additional layout features:

\begin{itemize}[leftmargin=\subparitemindent]
    \item \texttt{final} — (default) Compiles the document in its final version.
    \item \texttt{showframe} — Displays page layout frames (e.g., margins, headers, text block areas).
    \item \texttt{showgrid} — Displays a background grid to help visualize element positioning.
\end{itemize}

These options are useful during the drafting and formatting stages but should be disabled for the final submission. They are set in the document class declaration line in \texttt{main.tex}.
\end{subparagraph}

%========================
\subsection{Example Usage}
%========================
\begin{subparagraph}
The document class options are configured in the preamble of the \texttt{main.tex} file. Example configurations include:

\begin{itemize}[leftmargin=\subparitemindent]
    \item \texttt{\textbackslash documentclass[phd, ist]\{VISTEC\}}\\Ph.D. thesis, IST School (Information Science and Technology Program)
    \item \texttt{\textbackslash documentclass[master, mse]\{VISTEC\}}\\Master’s thesis, MSE School (Materials Science and Engineering Program)
    \item \texttt{\textbackslash documentclass[phd, ese, showframe]\{VISTEC\}}\\Ph.D. thesis, ESE School (Chemical Engineering Program) with layout frames displayed
\end{itemize}

\end{subparagraph}

%========================
\section{Editing Thesis Metadata}
%========================
\begin{paragraph}
Edit \texttt{thesisinfo.tex} to set your thesis title, author name, student ID, academic year, advisor, committee members, and program information. These metadata fields automatically populate the title page, approval page, and other formal sections.
\end{paragraph}

%========================
\section{Adding Content to Chapters}
%========================
\begin{paragraph}
Each main chapter (e.g., Introduction, Background, Investigation, Conclusion) should be placed under \texttt{contents/} and included using \verb|\include{}| in \texttt{main.tex}. You can create additional chapter files following the provided structure, and organize sections, figures, tables, algorithms, and citations inside them.
\end{paragraph}

%========================
\section{How to Use {\LaTeX}}
%========================
\begin{paragraph}
If you are new to {\LaTeX}, it is recommended to start with basic tutorials to understand fundamental concepts such as document structure, commands, environments, and referencing. A good starting point is the Overleaf online guide available at:

\begin{center}
\href{https://www.overleaf.com/learn}{\texttt{https://www.overleaf.com/learn}}
\end{center}

The Overleaf Learn platform provides comprehensive, beginner-friendly resources covering topics from basic document setup to advanced formatting and bibliography management. Familiarity with these concepts will significantly improve your ability to customize and work efficiently with this thesis template.
\end{paragraph}

%========================
\section{Compiling the Thesis}
%========================
\begin{paragraph}
Use \texttt{pdfLaTeX} as the compiler. A typical compilation sequence includes:
\begin{itemize}[leftmargin=\paritemindent]
    \item First, run \texttt{pdflatex main.tex} to generate auxiliary files.
    \item Then, run \texttt{bibtex main} to generate the bibliography.
    \item Finally, run \texttt{pdflatex main.tex} twice to resolve cross-references.
\end{itemize}

Alternatively, tools like \texttt{latexmk} or IDEs such as Overleaf, TeXShop, and VS Code with {\LaTeX} Workshop can automate this process.
\end{paragraph}